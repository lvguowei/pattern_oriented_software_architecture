% Created 2015-04-12 Sun 15:29
\documentclass{article}
\usepackage[utf8]{inputenc}
\usepackage[T1]{fontenc}
\usepackage{fixltx2e}
\usepackage{graphicx}
\usepackage{longtable}
\usepackage{float}
\usepackage{wrapfig}
\usepackage{rotating}
\usepackage[normalem]{ulem}
\usepackage{amsmath}
\usepackage{textcomp}
\usepackage{marvosym}
\usepackage{wasysym}
\usepackage{amssymb}
\usepackage{hyperref}
\tolerance=1000
\author{Guowei Lv}
\date{\today}
\title{Factory Pattern}
\hypersetup{
  pdfkeywords={},
  pdfsubject={},
  pdfcreator={Emacs 24.4.1 (Org mode 8.2.10)}}
\begin{document}

\maketitle
\tableofcontents


\section{Definition}
\label{sec-1}
The Factory Method Pattern defines an interface for creating an object, but lets subclasses decide which class to instantiate. Factory Method lets a class defer instantiation to subclasses.

\section{Example}
\label{sec-2}

\subsection{A framework for the pizza store}
\label{sec-2-1}

This is the abstract PizzaStore class:
\begin{verbatim}
package com.factory;
public abstract class PizzaStore {
    public Pizza orderPizza(String type) {
	Pizza pizza;
	pizza = createPizza(type);
	pizza.prepare();
	pizza.bake();
	pizza.cut();
	pizza.box();
	return pizza;
    }

    abstract Pizza createPizza(String type);
}
\end{verbatim}

Here are some concrete subclasses:

\begin{verbatim}
package com.factory;
public class NYPizzaStore extends PizzaStore {
  Pizza createPizza(String item) {
    if (item.equals("cheese")) {
      return new NYStyleCheesePizza();
    } else {
      return null;
    }
  }
}
\end{verbatim}

\begin{verbatim}
package com.factory;
public class ChicagoPizzaStore extends PizzaStore {
  Pizza createPizza(String item) {
    if (item.equals("cheese")) {
      return new ChicagoStyleCheesePizza();
    } else {
      return null;
    }
  }
}
\end{verbatim}


Here is the Pizza class:

\begin{verbatim}
package com.factory;

import java.util.ArrayList;

public abstract class Pizza {
  String name;
  String dough;
  String sauce;
  ArrayList toppings = new ArrayList();

  void prepare() {
    System.out.println("Preparing " + name);
    System.out.println("Tossing dough...");
    System.out.println("Adding sauce...");
    System.out.println("Adding toppings: ");
    for (int i = 0; i < toppings.size(); i++) {
      System.out.println("    " + toppings.get(i));
    }
  }

  void bake() {
    System.out.println("Bake for 25 minutes at 350");
  }

  void cut() {
    System.out.println("Cutting the pizza into diagnal slices");
  }

  void box() {
    System.out.println("Place pizza in official PizzaStore box");
  }

  public String getName() {
    return name;
  }
}
\end{verbatim}

Now we just need some concrete Pizza class:

\begin{verbatim}
package com.factory;
public class NYStyleCheesePizza extends Pizza {
  public NYStyleCheesePizza() {
    name = "NY Style Sauce and Cheese Pizza";
    dough = "Thin Crust Dough";
    sauce = "Marinara Sauce";

    toppings.add("Grated Reggiano Cheese");
  }
}
\end{verbatim}

\begin{verbatim}
package com.factory;
public class ChicagoStyleCheesePizza extends Pizza {
  public ChicagoStyleCheesePizza() {
    name = "Chicago Style Deep Dish Cheese Pizza";
    dough = "Extra Thick Crust Dough";
    sauce = "Plum Tomato Sauce";

    toppings.add("Shredded Mozzarella Cheese");
  }

  void cut() {
    System.out.println("Cutting the pizza into square slices");
  }
}
\end{verbatim}

The main class:

\begin{verbatim}
package com.factory;

public class Main {
  public static void main(String[] args) {
    PizzaStore nyStore = new NYPizzaStore();
    PizzaStore chicagoStore = new ChicagoPizzaStore();

    Pizza pizza = nyStore.orderPizza("cheese");
    System.out.println("Ethan ordered a " + pizza.getName() + "\n");

    pizza = chicagoStore.orderPizza("cheese");
    System.out.println("Joel ordered a " + pizza.getName() + "\n");
  }
}
\end{verbatim}

\section{The Gradle Build Script}
\label{sec-3}
\section{How to run}
\label{sec-4}
Press \texttt{C-c C-c} on the code block below.
% Emacs 24.4.1 (Org mode 8.2.10)
\end{document}
